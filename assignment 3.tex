\documentclass[twocolumn,12pt]{article}

\usepackage[utf8]{inputenc}
\usepackage{graphicx}
\usepackage{amssymb}
\usepackage{amsmath}
\title{Assignment 3}
\author{Gollapudi Sasank CS21BTECH11019}

\begin{document}
\maketitle
\section*{Question : }
The blood groups of 30 students of class VIII are recorded as follows :

\begin{center}
A,B,O,O,AB,O,A,O,B,A,O,B,A,O,O,\\
A,AB,O,A,A,O,O,AB,B,A,O,B,A,B,O
\end{center}

\noindent Represent this data in the form of a frequency distribution table. Which is the most common, and which is the rarest, blood group among these students? 
\section*{Solution : }
\begin{table}[ht!]
\begin{tabular}{|c|c|}

\hline
\textbf{Blood Group} & \textbf{Frequency} \\
\hline
O & $12$ \\
\hline
A & $9$ \\
\hline
B & $6$ \\
\hline
AB & $3$  \\
\hline
\end{tabular}
\centering
\caption{}
\label{table:table 1}
\end{table}
\noindent From the above table blood group \textbf{O} has the maximum frequency. So \textbf{O} is the most common blood group. Also blood group \textbf{AB} has the least frequency. So \textbf{AB} is the rarest blood group.\\ 

\end{document}